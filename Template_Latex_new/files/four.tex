%% Copyright 1998 Pepe Kubon
%%
%% `two.tex' --- 2nd chapter for thes-full.tex, thes-short-tex from
%%               the `csthesis' bundle
%%
%% You are allowed to distribute this file together with all files
%% mentioned in READ.ME.
%%
%% You are not allowed to modify its contents.
%%

%%%%%%%%%%%%%%%%%%%%%%%%%%%%%%%%%%%%%%%%%%%%%%%%%
%
%     Chapter 2   
%
%%%%%%%%%%%%%%%%%%%%%%%%%%%%%%%%%%%%%%%%%%%%%%%%

\chapter{introduction}
\label{four}
\textbf{motivation , primarily problem to solve}\newline
we need jobs to match skills we need to detect skills\newline
\textbf{real motivation secondary problem that we are actually solving}\newline
how to detect skills. why it is challenging and different than usual keyword extraction problems 


\section{literature review}

\section{what are the algorithms in keyword extraction }
\textbf{NLP introduction} 
The automatic analysis of 

text involves a deep understanding of 

natural language by machines, a reality 

from which we are still very far off \cite{review on NLP}.
Hitherto, online information 

retrieval, aggregation, and processing 

have mainly been based on algorithms 

relying on the textual representation of 

web pages. Such algorithms are very 

good at retrieving texts, splitting them 

into parts, checking the spelling and 

counting the number of words. When 

it comes to interpreting sentences and 

extracting meaningful information, however, their capabilities are known to 

be very limited. NLp in fact, requires high level symbolic capabilities(Dyer 1994) including:

1-creation and propagation of dynamic 

bindings;
2-manipulation of recursive, constituent structures,
 acquisition and access of lexical, semantic, and episodic memories;
❏ control of multiple learning/process- ing modules and routing of informa- tion among such modules;
❏ grounding of basic-level language constructs (e.g., objects and actions) in perceptual/motor experiences;
❏ representation of abstract concepts.
All such capabilities are required to shift from mere NLP to what is usually referred to as natural language under- standing (Allen, 1987). Today, most of the existing approaches are still based on the syntactic representation of text, a method that relies mainly on word co- occurrence frequencies. Such algorithms are limited by the fact that they can pro- cess only the information that they can ‘see’. As human text processors, we do not have such limitations as every word we see activates a cascade of semantically related concepts, relevant episodes, and sensory experiences, all of which enable the completion of complex NLP tasks—such as word-sense disam- biguation, textual entailment, and semantic role labeling—in a quick and
effortless way.\cite{review on NLP}


Tagging is the process of labeling web resources based on their content. Each label, or tag, corresponds to a topic in a given document. Unlike metadata assigned by authors, or by professional indexers in libraries, tags are assigned by end- users for organizing and sharing information that is of interest to them. The organic system of tags assigned by all users of a given web platform is called a folksonomy. \cite{folksonomy}

\textbf{what is keyphrase? what is keyphrase extraction?what is the goal of extracting keyphrases?}
Many journals ask their authors to provide a list of keywords for their articles. We call these keyphrases, rather than keywords, because they are often phrases of two or more words, rather than single words. We define a keyphrase list as a short list of phrases (typically five to fifteen noun phrases) that capture the main topics discussed in a given document. This paper is concerned with the automatic extraction of keyphrases from text.
Keyphrases are meant to serve multiple goals. For example, (1) when they are printed on the first page of a journal article, the goal is summarization. They enable the reader to quickly determine whether the given article is in the reader’s fields of interest. (2) When they are printed in the cumulative index for a journal, the goal is indexing. They enable the reader to quickly find a relevant article when the reader has a specific need. (3) When a search engine form has a field labelled keywords, the goal is to enable the reader to make the search more precise. A search for documents that match a given query term in the keyword field will yield a smaller, higher quality list of hits than a search for the same term in the full text of the documents. Keyphrases can serve these diverse goals and others, because the goals share the requirement for a short list of phrases that captures the main topics of the documents.
We define automatic keyphrase extraction as the automatic selection of important, topical phrases from within the body of a document. Automatic keyphrase extraction is a special case of the more general task of automatic keyphrase generation, in which the generated phrases do not necessarily appear in the body of the given document. 
\textbf{what is the differences between indexing and keyphrase list}

\textbf{supervised} \newline
\textbf{unsupervised}\newline
\textbf{graph-based algorithms}\newline
\textbf{what works and what fails}\newline
\textbf {Automatic keyphrase extraction}
\section{our approach our algorithm}
\textbf{our algorithm a bit of introduction}\newline
\textbf{data acquisition}\newline
\textbf{data cleaning}\newline
\textbf{the results}\newline

\section{visualization of trends}

\section{conclusion}


